\documentclass{ctexart}
\usepackage{fancyhdr}
\usepackage{graphicx}
\usepackage{lipsum}
\usepackage{calc}
\usepackage{array}
\usepackage[margin=2.5cm, headsep=1cm, headheight=87pt]{geometry}
\usepackage{array}
\usepackage{amsmath}
\usepackage{amsfonts}
\usepackage{tikz}
\usepackage{booktabs}
\usepackage{longtable}
\usepackage{tabularx}
\usetikzlibrary{positioning}

\renewcommand{\arraystretch}{1.2}
\setlength{\extrarowheight}{5pt}

\pagestyle{plain}
\fancyhf{}

\begin{document}


 
 \includegraphics[width=0.9\textwidth]{image.png} % 确保图片文件存在
 \vspace{1cm} % 图片与标题之间的垂直间距
 {\centering 
 
 \Huge \bfseries 面向5G网络切片的异构蜂窝网络无线资源与能效联合优化方案设计 \par}
 \vspace{2cm} % 调整这里来控制标题和正文之间的距离

\begin{abstract}
\noindent
本文面向 5G 异构蜂窝网络的切片化资源调度,围绕“QoS 最大化”与“能耗最小化”的综合目标,构建由浅入深的五层模型:
(1)单微基站、单周期的静态 RB 划分;
(2)多周期、随时间演化的动态分配;
(3)多微基站频谱复用下的资源—功率联合优化;
(4)宏—微协同的接入选择与切片划分;
(5)在(4)基础上引入能耗模型的 QoS–能效多目标优化。

\medskip\noindent
\textbf{方法:}问题(1)建模为带整除约束的整数规划,小规模可行域枚举给出全局最优;(2)引入任务排队与信道时变,将阶段性决策表述为滚动时域优化,采用启发式 Beam Search 与邻域交换求近优;(3)在干扰模型下,将每基站同切片等功率的假设纳入决策,形成 RB–功率的分层协调:外层按切片与基站分配 RB,内层在干扰感知下搜索功率;(4)加入宏基站与接入选择,采用“接入—切片—功率”三级联动;(5)以能耗为罚项(权重 $\mu$)转化为单目标,沿用分层框架并作帕累托分析。

\medskip\noindent
\textbf{结果:}(1)单基站最优分配为 $(20,20,10)$(URLLC,eMBB,mMTC);(2)在 10 次滚动决策中,动态方案较静态基线的总效用提升约 $x\%$,URLLC 丢包率显著降低;(3)多基站功率控制有效抑制复用干扰,跨用户的速率达标率提升;(4)宏—微协同提升总体 QoS,减少边缘用户的拥塞;(5)$\mu\in[10,30]$ 下存在稳定的帕累托前沿,取 $\mu=20$ 可在 QoS 与能耗间获得均衡。

\medskip\noindent
\textbf{结论:}效用折扣系数与到达强度是影响 QoS 的主要因素;功率上限与复用密度决定干扰抑制的收益。所提“分层—滚动—功率协调”范式在切片化、时变信道与多点干扰的综合条件下实现了稳定的近最优决策,可为 5G/6G 无线资源编排与绿色运营提供参考。

\end{abstract}

\noindent\textbf{关键词:} 束搜索;能效优化;功率控制;QoS;5G


\section{问题重述}

\subsection{问题背景}
随着物联网设备的指数级增长,移动通信需求呈现爆发式增长,推动了网络架构的不断演进。为应对不同场景下的数据密度与服务需求,移动通信网络逐步采用宏基站(MBS)与小基站(SBS)混合部署的异构蜂窝网络结构。在这种异构网络中,5G引入了网络切片技术,采用网络功能虚拟化将物理网络划分为多个逻辑独立的切片,如高可靠低时延切片(URLLC)、增强移动宽带切片(eMBB)和大规模机器通信切片(mMTC)。每个切片可为特定的应用场景或用户需求提供定制化的网络服务,并需满足各自的服务水平协议(SLA),包括最低速率和最大延迟等要求。

无线资源采用正交频分多址(OFDMA)技术,在时域和频域上被划分为多个资源块(RB),并将这些资源块灵活分配给所有的切片。在长时间的通信服务过程中,用户任务以一定的概率到达,且用户有移动性,导致信道状况动态变化,因此系统需周期性地重新资源配置。此外,在多个微基站共存的场景下,因频率复用,不同传输链路之间可能存在干扰。基站的发射功率控制是降低干扰、提升用户体验的关键。同时,运营商的能耗控制也是一项核心考量,需在保障服务质量的同时,采用功率优化降低能耗以减少运营成本。面对多切片共存、异构网络协同的复杂场景,如何高效分配无线资源成为保障网络性能与服务质量的关键问题。

\subsection{问题提出}
基于上述背景,竞赛要求解决以下五个问题:

\begin{enumerate}
 \item 问题一:针对单个微基站向其覆盖范围内的一个用户提供服务,该基站拥有50个资源块,需将所有资源块分配给三类切片,以使用户服务质量达最大。
 \item 问题二:在用户在一段时间内以概率多次传输请求并移动的动态场景下,系统需对资源10次决策。要求给出每次决策的三类切片资源块最佳分配方案,使整体用户服务质量达最大,同时需考虑排队队列中的积压任务。
 \item 问题三:针对多个微基站存在干扰的情况,要求给出各基站每次决策的资源块分配方案和发射功率控制方案,使系统的用户服务质量达最大。
 \item 问题四:在宏基站和多个微基站组成的异构网络中,需确定每个用户的接入决策,并给出各基站的切片划分决策和发射功率分配方案,以实现最大用户服务质量。
 \item 问题五:在问题四的模型基础上,以运营商能耗为新的优化目标,要求给出合理的资源分配策略,使能耗最低的同时能达最大的用户服务质量。
\end{enumerate}

\section{问题分析}
\begin{figure}[htbp]
 \centering
 \begin{tikzpicture}[
 node style/.style = {rectangle, rounded corners, draw=black, minimum width=6cm, minimum height=1.5cm, align=center, font=\large\bfseries},
 arrow style/.style = {->, thick},
 label style/.style = {font=\small},
 x_dist = 8cm,
 y_dist = 3cm
 ]

 % 问题一
 \node (q1) at (0,0) {\textbf{问题一}};
 \node[node style, right=of q1, xshift=2cm] (a1) {单个用户,单个基站};
 \draw[arrow style] (q1) -- (a1);

 % 问题二
 \node (q2) [below=of q1] {\textbf{问题二}};
 \node[node style, right=of q2, xshift=2cm] (a2) {多个决策周期,动态用户};
 \draw[arrow style] (q2) -- (a2);
 \draw[arrow style] (q1) -- node[label style, right, yshift=0cm, xshift=0cm] {添加决策数量} (q2);

 % 问题三
 \node (q3) [below=of q2] {\textbf{问题三}};
 \node[node style, right=of q3, xshift=2cm] (a3) {多个基站,干扰,功率控制};
 \draw[arrow style] (q3) -- (a3);
 \draw[arrow style] (q2) -- node[label style, right, yshift=0cm, xshift=0cm] {添加多个基站和干扰} (q3);
 
 % 问题四
 \node (q4) [below=of q3] {\textbf{问题四}};
 \node[node style, right=of q4, xshift=2cm] (a4) {宏微混合网络,联合优化};
 \draw[arrow style] (q4) -- (a4);
 \draw[arrow style] (q3) -- node[label style, right, yshift=0cm, xshift=0cm] {添加宏基站,异构网络} (q4);
 
 % 问题五
 \node (q5) [below=of q4] {\textbf{问题五}};
 \node[node style, right=of q5, xshift=2cm] (a5) {能耗与QoS联合优化};
 \draw[arrow style] (q5) -- (a5);
 \draw[arrow style] (q4) -- node[label style, right, yshift=0cm, xshift=0cm] {添加能耗目标} (q5);

 \end{tikzpicture}
 \caption{解题思路流程图}
 \label{fig:solution_flowchart}
\end{figure}

\subsection{问题一分析}
问题一是一个整数规划问题,要求在单一基站的场景下,对该基站的50个资源块(RB)分配,以最大化用户的服务质量(QoS)。每个用户只到达了一个任务,分配对象是三类网络切片:高可靠低时延切片(URLLC)、增强移动宽带切片(eMBB)和大规模机器通信切片(mMTC)。用户的任务为下行传输,且三种切片占用的资源块数应小于该基站资源块总数。

\subsection{问题二分析}
问题二在问题一的基础上,考虑用户在一段时间内以一定概率传输请求并移动的动态场景。系统需在每个决策周期后对资源重新分配,以实现下一个决策周期的最大用户服务质量,同时需考虑排队队列中的积压任务,所以决策过程必须考虑时间连续性及当前决策对未来状态的影响。基于问题一的整数规划模型,问题二采用启发式算法降低搜索的时间复杂度,采用束搜索只维护一个固定大小的解集,再从该解集出发寻找下一轮的所有可能解,利用贪心算法的思想获取解集中的最优解作为下一次搜索的出发点,避免搜索空间的爆炸性增长,使在有限的计算资源下仍找到高质量解决方案。

\subsection{问题三分析}
问题三是问题一和问题二的扩展,引入了多微基站(SBS)和信号干扰的复杂性。核心挑战在于需同时资源块(RB)分配和发射功率控制,以最大化所有用户的总服务质量(QoS)。用户通常会在一段时间内以概率多次传输请求并移动的动态场景。因微基站之间存在频率复用,不同传输链路会相互干扰,这使信干噪比(SINR)的计算变得更为复杂。

为应对这种高复杂度的联合优化问题,本文可利用题目的特性来简化模型。根据附件信息,三种切片的任务到达分布(泊松分布、均匀分布)都是无记忆的。基于这一特性,本文可做出一个合理假设:系统的最优资源分配和功率配置在相邻决策周期内无需发生剧烈波动。这为本文提供了一个解耦联合优化问题的启发式思路。

本题可采用一种分层优化的策略:将发射功率视为一个超参数调优,在每个固定的功率值下,求解资源块的分配问题,从而将原本复杂的联合优化问题分解为两个嵌套的子问题。

\textbf{1. 功率调优(外层问题)}:本文将发射功率 $P$ 视为超参数,在其离散取值范围 [10, 30] dBm 内调优。采用遍历或有限的启发式搜索来找到能使系统总QoS达最大化的最优功率值。

\textbf{2. 资源分配(内层问题)}:对每个固定的功率值,资源分配问题成为一个动态规划问题,类似于问题二。本文可采用束搜索算法来解决这一问题,它能应对用户任务的动态到达和移动性。在资源分配过程中,用户的接入决策也至关重要,用户会根据当前信道质量(SINR)等信息,选择接入最适合的基站。

这种分层求解策略在保证计算效率的同时,能有效解决资源分配、功率控制和干扰管理问题,从而在动态多基站场景下找到一个高质量的近似最优解。

\subsection{问题四分析}
问题四是问题三的进一步扩展,将网络拓扑从纯微基站(SBS)网络升级为宏微混合的异构蜂窝网络。这引入了新的复杂性,但也带来了一些简化。核心挑战在于需对用户的接入基站、切片资源分配和基站发射功率联合优化,以最大化系统的总服务质量(QoS)。
网络中存在一个宏基站(MBS)和多个微基站(SBS)协同服务用户 。MBS的频谱资源更充裕(100个资源块),且发射功率范围更高([10, 40] dBm) 。SBSs的资源块数量为50,功率范围为 [10, 30] dBm 。MBS和SBSs的频谱不重叠,因此两类基站之间不存在相互干扰 。但,微基站之间因频率复用而产生的干扰依然存在 。

问题四是一个高度复杂的联合优化问题,与问题三基本相同,区别只是添加了一个宏基站,而宏基站的资源块数量和发射功率范围更大,因此可视为一个特殊的微基站,而它的频率不会对其他基站产生影响,所以其余的基站中用户的信干燥比的计算方法不变,只是需单独计算宏基站的Qos使其最大即可。本文可采用类似于问题三的分层求解策略继续求解。

\subsection{问题五分析}
问题五需在问题四的基础上,进一步考虑能耗问题,即在最大化用户服务质量(QoS)的同时,降低系统的能耗。能耗主要包括基站的固定能耗,发射能耗和资源块的激活能耗。每个基站的发射功率控制和资源块分配都将直接影响系统的总能耗,资源块的分配又将影响用户服务质量,所以需在问题四目标函数的基础上构造一个罚函数,使新的目标函数兼具提高用户服务质量和降低基站能耗的功能。
\section{模型假设}

为便于问题的研究,对题目中某些条件简化及合理的假设: 
\begin{itemize}
 \item \textbf{假设一}:题目中所有的宏基站均持有100个资源块,可在[10, 40] dBm范围内调节发射功率;微基站均持有50个资源块,可在[10, 30] dBm范围内调节发射功率。宏基站和微基站之间互相不干扰。
 \item \textbf{假设二}:每类切片的资源块数必须为每个用户的资源块占用量的整数倍
 \item \textbf{假设三}:因每个用户的任务到达分布(泊松分布、均匀分布)都是无记忆的,系统的最优资源分配和功率配置在相邻决策周期内无剧烈波动。
 \item \textbf{假设四}:用户接入距离自己最短的基站。
 \item \textbf{假设五}:每一个决策周期用户只接入一个基站。
 \item \textbf{假设六}:每个基站的发射功率和资源块分配是独立的。
 \item \textbf{假设七}:功率的正整数情况可很好地接近最优解。
\end{itemize}


\section{主要符号说明}
如下表所示:

\begin{table}[htbp]
 \centering
 \caption{符号说明}
 \label{tab:symbols}
 \resizebox{0.9\textwidth}{!}{%
 \begin{tabular}{l p{0.5\textwidth} l l p{0.5\textwidth} l}
 \toprule
 \textbf{符号} & \textbf{含义} & \textbf{单位} & \textbf{符号} & \textbf{含义} & \textbf{单位} \\
 \midrule

 $P_{n,k}$ & 基站 $n$ 到用户$k$ 的发射功率 & $dBm$&
 $\phi_{n,k}$ & 用户$k$ 与基站 $n$ 之间由路径损耗造成的的大规模衰减 & $dB$ \\
 $h_{n,k}$ & 用户$k$ 与基站 $n$ 之间由信号在空间中多径效应造成的小规模瑞丽衰减 & &
 $N_0$ & 白噪声 & \\
 $i$ & 用户持有的资源块数量 &  &
 $b$ & 单个资源块的带宽 &  \\
 $\gamma $ & 用户的信噪比 & &
 $P_{rx}$ & 接收信号功率 & $mW$ \\
 $r$ & 用户的传输速率 &  &
 $y^{URLLC}$ &  URLLC 用户任务的服务质量 &  \\
 $y^{eMBB}$ &  eMBB 用户任务的服务质量 &  &
 $y^{mMTC}$ &  mMTC 用户任务的服务质量 &  \\
 $\alpha$ & URLLC 用户任务效用折扣系数 &  &
 $L_{SLA}$ & URLLC,eMBB,mMTC的延迟服务水平协议 & $ms$ \\
 $r_{SLA}$ & eMBB 切片的传输速率服务水平协议  & $Mbps$ &
 $M^{URLLC}$ &  URLLC 用户任务的惩罚系数 & \\
 $M^{eMBB}$ & eMBB 用户任务的惩罚系数 & &
 $M^{mMTC}$ & mMTC 用户任务的惩罚系数 &  \\
 $P_{static}$ & 固定能耗 & $W$ &
 $P_{RB}$ & 每个启动的资源块产生的额外功耗 & $W$ \\
 $\delta $ & 激活能耗系数 &  &
 $N_{active}$ & 资源块启动数量 & \\
 $P_{tx}$ & 因功率放大器会对信号强度造成一定的损耗的发射功率 & $W$ \\
 $\eta $ & 损耗系数 & &
 $P_{transmit}$ & 发射功率 & $W$ \\
 $P$ & 总能耗 & $W$ & & &\\\bottomrule
 \end{tabular}}
\end{table}


\section{模型建立与求解}
\subsection{问题一模型建立与求解}
问题一旨在为一个用户在单个微基站场景下分配无线资源,以最大化其服务质量(QoS)。该问题可建模为一个离散优化问题,其目标函数和约束条件如下:

\subsubsection{目标函数与约束条件的建立}

最大化用户总体服务质量。用户总体服务质量可表示为分配给三类切片的效用值之和:

$$\max \quad Y_{total} = \sum_{k \in U} y_k^{URLLC} + \sum_{k \in E} y_k^{eMBB} + \sum_{k \in M} y_k^{mMTC}$$

其中:
\begin{itemize}
 \item $U$、 $E$ 和 $M$ 分别表示 URLLC、eMBB 和 mMTC 三类切片中的用户集合。
 \item $y_k^{URLLC}$ 是 URLLC 切片中用户 $k$ 的服务质量函数。
 \item $y_k^{eMBB}$ 是 eMBB 切片中用户 $k$ 的服务质量函数。
 \item $y_k^{mMTC}$ 是 mMTC 切片中用户 $k$ 的服务质量函数[3]。
\end{itemize}

\textbf{1.1 决策变量}

决策变量为分配给三类切片的资源块(RB)数量,记为 $n_{URLLC}$、$n_{eMBB}$ 和 $n_{mMTC}$。

\textbf{1.2 约束条件}

1. \textbf{资源块总数约束}:被分配的资源块的数量之和不多于50。
 $$n_{URLLC} + n_{eMBB} + n_{mMTC} \le 50$$

2. \textbf{资源块单位约束}:每类切片每个用户任务的资源块占用量是固定的。
 
 \begin{itemize}
 \item URLLC:每个任务需占用10个RB。
 \item eMBB:每个任务需占用5个RB。
 \item mMTC:每个任务需占用2个RB。
 \end{itemize}
 
 因此,$n_{URLLC}$ 必须是10的非负整数倍,$n_{eMBB}$ 是5的非负整数倍,$n_{mMTC}$ 是2的非负整数倍。
 


3. \textbf{变量类型约束}:决策变量为非负整数。
 $$n_{URLLC}, n_{eMBB}, n_{mMTC} \in \mathbb{N}_{\ge 0}$$

\textbf{1.3 效用函数计算}

为计算每类切片的服务质量,需首先确定传输速率和总延迟。

\textbf{1.3.1 传输速率 $r$ 和延迟 $L$ 的计算}

\textbf{1.3.1.1 传输速率 $r$ 的计算}

用户的传输速率 $r$ 是基于信干噪比 $\gamma$ 和资源块数量 $i$ 采用香农公式计算的:
$$r=ib \cdot \log_2(1+\gamma)$$
其中,$i$ 为分配给用户的资源块数量,$b$ 为单个资源块的带宽。信干噪比 $\gamma$ 可采用接收信号功率 $p_{rx}$ 和白噪声 $N_0$ 计算得到。

\textbf{白噪声 $N_0$}:
白噪声 $N_0$ 与用户所持有的资源块数量 $i$ 和单个资源块的带宽 $b$ 成正比,同时包含热噪声谱密度和接收机的噪声系数 $NF$。其计算公式为:
$$N_{0}=-174+10\log_{10}(ib)+NF$$

\textbf{接收信号功率 $p_{rx}$}:
接收信号功率 $p_{rx}$ 的单位为mW,可采用发射功率 $p_{n,k}$、大规模瑞丽衰减 $\phi_{n,k}$ 和小规模瑞丽衰减 $|h_{n,k}|^2$ 计算得到:
$$p_{rx}=10^{\frac{p_{n,k}-\phi_{n,k}}{10}}\cdot|h_{n,k}|^{2}$$
其中,$p_{n,k}$ 的单位为dBm,$\phi_{n,k}$ 的单位为dB,而 $|h_{n,k}|^2$ 无量纲。

\textbf{信干噪比 $\gamma$}:
在多基站场景下,信干噪比 $\gamma$ 可表示为:
$$\gamma=\frac{p_{n,k}\phi_{n,k}|h_{n,k}|^{2}}{\sum_{u\in N,u\ne n}p_{u,k}\phi_{u,k}|h_{u,k}|^{2}+N_{0}}$$
其中,分子部分为接收信号功率,分母部分为干扰信号功率与白噪声之和[1]。对问题一的单基站场景,干扰项 $\sum_{u\in N,u\ne n}p_{u,k}\phi_{u,k}|h_{u,k}|^{2}$ 为0。

\textbf{1.3.1.2 总延迟 $L$ 的计算}:
总延迟 $L$ 由排队延迟 $Q$ 和传输延迟 $T$ 组成,即 $L=Q+T$。
\begin{itemize}
 \item 排队延迟 $Q$:该延迟取决于任务队列中等待传输的任务数量。
 \item 传输延迟 $T$:传输延迟可由任务数据量除以用户的传输速率 $r$ 得到 [2]。$$T=\frac{\text{任务数据量}}{r}$$
\end{itemize}


\textbf{1.3.2 服务质量函数}

* \textbf{高可靠低时延切片(URLLC)}
 $$
 y^{URLLC}=\begin{cases}\alpha^{L}&L\le L_{SLA}\\ -M^{URLLC}&L>L_{SLA}\end{cases}
 $$
 其中,$\alpha$ 为 URLLC 用户任务效用折扣系数,$\alpha\in (0,1)$。$L_{SLA}$ 是用户任务能忍受的最大延迟,$-M^{URLLC}$ 为任务损失的惩罚值。

* \textbf{增强移动宽带切片(eMBB)}
 $$
 y^{eMBB}=\begin{cases}1&r\ge r_{SLA}\text{ and }L\le L_{SLA}\\ r/r_{SLA}&r<r_{SLA}\text{ and }L\le L_{SLA}\\ -M^{eMBB}&L>L_{SLA}\end{cases}
 $$
 其中,$r_{SLA}$ 是 eMBB 切片的传输速率 SLA,$-M^{eMBB}$ 为针对 eMBB 用户任务损失的惩罚值。

* \textbf{大规模机器通信切片(mMTC)}
 $$
 y^{mMTC}=\begin{cases}\frac{\sum_{i\in J}c_{i}}{\sum_{i\in J}C_{i}}&L\le L_{SLA}\\ -M^{mMTC}&L>L_{SLA}\end{cases}
 $$
 其中,$C_i$ 是一个二元变量,表示在某个决策周期内用户 $i$ 是否有任务需上传。$c_i$ 是一个二元变量,表示用户 $i$ 是否在该周期内成功接入。$-M^{mMTC}$ 为任务损失的惩罚值。

\subsubsection{求解方法}
本问题是一个离散且有明确约束的优化问题。因决策变量的取值范围较小,可采用\textbf{穷举法(或枚举法)}来寻找最优解。具体步骤如下:
\begin{enumerate}
 \item \textbf{枚举所有可行解}:根据约束条件,列举所有满足总资源块为50,且分配数量为10、5、2倍数的组合 $(n_{URLLC}, n_{eMBB}, n_{mMTC})$,可行解数量不超过$|U|*|E|*|M|=80$。
 \item \textbf{计算各组合下的总服务质量}:对每一个可行解,根据上述模型和参数,计算其对应的传输速率、延迟,并进一步计算出每类切片的效用值,最终得到总体用户服务质量 $Y_{total}$。
 \item \textbf{确定最优解}:比较所有可行解下的总服务质量,选取使 $Y_{total}$ 达最大值的分配方案作为问题的最优解。
\end{enumerate}

\subsubsection{结果分析}
为在单一基站、50个资源块(RB)的约束下最大化用户服务质量(QoS),模型找到了一个最优的资源分配方案,使总服务质量达了15.859。

\begin{itemize}
 \item \textbf{资源分配方案}:
 模型的最终决策为URLLC、eMBB和mMTC三类切片分别分配了20 RB、20 RB和10 RB。这一分配方案成功地为 URLLC 服务了2个任务,为 eMBB 服务了4个任务,为 mMTC 服务了10个任务。资源块分配决策和用户服务质量大小如下表所示:

 \begin{table}
\centering
\begin{tabular}{lllll} 
\hline
 & URLLC & cMBB & mMTC & total \\ 
\hline
资源块(RB) & 20 & 20 & 10 & 50 \\
用户服务质量(Y) & 1.989 & 3.870 & 11 & 15.859 \\
\hline
\end{tabular}
\end{table}
 \newpage
 \item \textbf{各切片对QoS的贡献}:
 从分片得分来看,mMTC切片以11的得分成为总QoS评分的主要贡献者,占比最高。URLLC和eMBB切片也分别贡献了1.989和3.870,共同构成了总服务质量。

 \item \textbf{分配策略分析}:
 这一分配方案体现了模型在满足约束的同时,对不同切片效用值的权衡。mMTC切片以较少的资源(10 RB)服务了最多的任务数(10个),获得了最高的总分,这表明在当前任务流和信道条件下,mMTC切片是提高总QoS的最优选择。eMBB和URLLC切片虽然占用了更多的资源,但其贡献值相对较低,这可能与它们的SLA要求更高或当前信道条件不佳有关。
\end{itemize}

总之,该结果展示了模型能基于不同切片的特性和SLA要求,合理的资源分配,以实现总服务质量的最大化。

\subsection{问题二模型建立和求解}
\subsubsection{模型建立}
该问题可建模为一个多阶段动态整数规划模型,其目标函数和约束条件如下:

\textbf{1.1 目标函数}
系统需在1000ms内10次决策。因此,目标函数是最大化在所有决策周期 $t$ 内的总服务质量:
$$\max \quad Y_{total} = \sum_{t=1}^{10} Y_{total}(t)$$
其中,$Y_{total}(t)$ 是第 $t$ 个决策周期内的总用户服务质量。

\textbf{1.2 决策变量}
决策变量为每个决策周期 $t$ 内分配给三类切片的资源块数量,记为 $n_{URLLC}(t)$、$n_{eMBB}(t)$ 和 $n_{mMTC}(t)$。

\textbf{1.3 约束条件}
\begin{enumerate}
 \item \textbf{资源块总数约束}:每个决策周期内,分配给三类切片的资源块总数不多于50。
 $$n_{URLLC}(t) + n_{eMBB}(t) + n_{mMTC}(t) \le 50, \quad \forall t \in \{1, \dots, 10\}$$
 \item \textbf{资源块单位约束}:分配给每类切片的资源块数量必须为该切片每个任务所需资源块数量的整数倍。
 $$n_{URLLC}(t) \in \{10k_1 | k_1 \in \mathbb{N}_{\ge 0}\} $$
 $$n_{eMBB}(t) \in \{5k_2 | k_2 \in \mathbb{N}_{\ge 0}\} $$
 $$n_{mMTC}(t) \in \{2k_3 | k_3 \in \mathbb{N}_{\ge 0}\} $$
 
 \item \textbf{任务队列约束}:每个决策周期开始时,需考虑排队队列中的积压任务。队列中的任务将优先被处理。
\end{enumerate}

\subsubsection{求解方法}
因该问题涉及多个决策周期和动态变化的用户状态,采用穷举法将变得非常复杂。本方案采用束搜索(Beam Search)与贪心算法相结合的方法来求解,这是一个有效的启发式(heuristic)搜索策略。

\textbf{一.束搜索}

 \textbf{原理}

束搜索是一种有限宽度的启发式图搜索算法,它采用在每一步搜索中保留少量“最优”节点来提高计算效率。该算法维护一个固定大小为 $W$ 的“束”(beam),其中 $W$ 为束宽。其核心思想是:在搜索的每一层,算法从当前束中的 $W$ 个节点出发,为每个节点扩展出 $B$ 个子节点($B$ 为分支因子),从而得到一个包含 $W \times B$ 个候选节点的集合。随后,算法利用一个启发式函数对所有候选节点评估和排序,并从中选出启发式函数值最优的前 $W$ 个节点作为下一层的束。该过程不断迭代,直至找到目标状态或无法继续扩展。

与广度优先搜索(BFS)等穷举算法相比,束搜索采用限制搜索广度极大地减少了计算复杂度和内存消耗。然而,因它在每一步都舍弃了部分节点,因此无法保证找到全局最优解。最优解可能存在于被剪枝的节点中,这体现了该算法以牺牲最优性为代价来换取高效率的特点。



下面是W=2,B=3的束搜索示意图:

\begin{document}
\begin{figure}[htbp]
 \centering
 \begin{tikzpicture}[
 promising/.style={circle, draw, fill=black, inner sep=4pt}, % 增大圆圈
 pruned/.style={circle, draw, inner sep=4pt, fill=white}, % 增大圆圈
 level 1/.style={sibling distance=3cm},
 level 2/.style={sibling distance=1.5cm},
 level 3/.style={sibling distance=1.5cm}
 ]

 % Define nodes
 \node[promising] (root) at (0,0) {};

 % Level 2
 \node[promising] (l2_1) at (-3,-2) {};
 \node[pruned] (l2_2) at (0,-2) {};
 \node[promising] (l2_3) at (3,-2) {};

 % Level 3
 \node[pruned] (l3_1) at (-4.5,-4) {};
 \node[promising] (l3_2) at (-3,-4) {};
 \node[pruned] (l3_3) at (-1.5,-4) {};
 \node[pruned] (l3_4) at (1.5,-4) {};
 \node[pruned] (l3_5) at (3,-4) {};
 \node[promising] (l3_6) at (4.5,-4) {};

 % Level 4
 \node[pruned] (l4_1) at (-4.5,-6) {};
 \node[promising] (l4_2) at (-3,-6) {};
 \node[pruned] (l4_3) at (-1.5,-6) {};
 \node[pruned] (l4_4) at (3.75,-6) {};
 \node[promising] (l4_5) at (5.25,-6) {};
 \node[pruned] (l4_6) at (6.75,-6) {};

 % Draw arrows
 \draw[->] (root) -- (l2_1);
 \draw[->] (root) -- (l2_2);
 \draw[->] (root) -- (l2_3);

 \draw[->] (l2_1) -- (l3_1);
 \draw[->] (l2_1) -- (l3_2);
 \draw[->] (l2_1) -- (l3_3);

 \draw[->] (l2_3) -- (l3_4);
 \draw[->] (l2_3) -- (l3_5);
 \draw[->] (l2_3) -- (l3_6);

 \draw[->] (l3_2) -- (l4_1);
 \draw[->] (l3_2) -- (l4_2);
 \draw[->] (l3_2) -- (l4_3);
 
 \draw[->] (l3_6) -- (l4_4);
 \draw[->] (l3_6) -- (l4_5);
 \draw[->] (l3_6) -- (l4_6);

 % Draw vertical dotted lines from the last nodes of each branch
 \draw[dashed, ->] (l4_2.south) -- +(0,-0.5);
 \draw[dashed, ->] (l4_5.south) -- +(0,-0.5);

 % Add text
 \node[align=center, text=green, below=1cm] at (0,-7) {继续搜索直到找到目标状态\\ 或无法继续前进};

 \end{tikzpicture}
 \caption{束搜索示意图}
 \label{fig:beam_search}
\end{figure}

\newpage{}

 \textbf{在本题中的应用}

在本题中,每一轮决策代表搜索图中的一个节点,而资源分配方案则是从一个节点到另一个节点的边。束搜索采用以下方式应用于本题:

1. 启发式函数:为评估每个资源分配方案的“前景”,本文设计一个启发式函数。该函数的目标是平衡当前的收益(QoS)和未来的潜在成本(积压任务)。本题采用的启发式函数为:
 $$H = \text{QoS} + \begin{cases} \frac{\lambda}{m} & m \ne 0 \\ {\varepsilon} & m = 0 \end{cases}$$
 其中:
 \begin{itemize}
 \item QoS 是当前决策周期所实现的用户服务质量,代表了当下的收益。
 \item 假设当前已$d$次决策,$m$ 是在进行 $d+1$ 次决策后,任务队列中积压的任务总量。$m$ 值越小,意味着未来状态越好。
 \item $\lambda$ 是一个超参数,用于平衡当前 QoS 和未来积压任务量的重要性。$\lambda$ 的值越大,算法在选择方案时越倾向于减少未来积压任务,即使这可能牺牲部分当前 QoS。
 \item $\varepsilon$ 是一个常量,仅在 $m=0$ 时加入。
 \end{itemize}

2. 算法流程:在每个决策周期,算法将生成若干个可行的资源分配方案。然后,利用Qos作为启发式函数对每个方案评分(上面的启发式函数在模型评估与敏感度分析中应用)。束搜索将根据评分从所有方案中选择出最优的 $W$ 个方案作为当前周期的解集。这个解集将作为下一个决策周期的搜索起点,如此循环,直到完成所有决策周期。最终,从最后一个周期的解集中选取总服务质量最高的方案作为最终结果。

\textbf{二.解题步骤}

\begin{enumerate}
 \item \textbf{初始化}:设置束宽$W$和束大小$B$,及启发式函数中的超参数$\lambda $。在第一个决策周期 $d=1$ 时,根据当前的任务队列情况,使用贪心策略来确定最佳的资源分配方案。
 \item \textbf{束搜索}:对每一个决策周期 $d$:
 \begin{itemize}
 \item \textbf{状态扩展}:根据当前周期 $d$ 的资源分配方案和用户任务到达情况,预测下一个周期 $d+1$ 可能的几种任务队列状态。
 \item \textbf{剪枝}:仅保留那些能带来最高短期回报(即最高用户服务质量)的几个“最优”资源分配方案作为解的集合,形成一个“束”(beam)。
 \item \textbf{重复}:在下一个周期 $d+1$ 中,从上一个周期的“束”中的每一个解开始,重复上述步骤,继续探索和扩展,并持续剪枝。
 \end{itemize}
 \item \textbf{最终决策}:经过10个决策周期后,从最终的“束”中选出总服务质量最高的路径,即为问题的最优解。
\end{enumerate}

\subsubsection{结果分析}
\textbf{1.结论}
利用上面提到的束搜索算法,本文对问题二了求解。以下是10次决策的最佳的资源块分配决策方案及对应的服务质量结果:
\begin{longtable}{cccccc}
 \caption{每一次的资源块分配决策及对应的服务质量} \\
 \toprule
 \textbf{决策轮数(t)} & \textbf{分配决策} & \textbf{$Y_{total}(t)$} & \textbf{$Y_{URLLC}(t)$} & \textbf{$Y_{eMBB}(t)$} & \textbf{$Y_{mMTC}(t)$} \\
 \midrule
 \endfirsthead

 \multicolumn{6}{c}%
 {\tablename\ \thetable\ -- 续表} \\
 \toprule
 \textbf{决策轮数(t)} & \textbf{分配决策} & \textbf{$Y_{total}(t)$} & \textbf{$Y_{URLLC}(t)$} & \textbf{$Y_{eMBB}(t)$} & \textbf{$Y_{mMTC}(t)$} \\
 \midrule
 \endhead

 \bottomrule
 \endlastfoot
 
 1 & (10, 5, 34) & 761.57 & 23.95 & 29.62 & 708.00 \\
 2 & (10, 5, 34) & 638.00 & 22.85 & -85.85 & 701.00 \\
 3 & (10, 5, 34) & 572.73 & 24.85 & -155.12 & 703.00 \\
 4 & (10, 5, 34) & 544.09 & 24.00 & -195.91 & 716.00 \\
 5 & (10, 10, 30) & 878.71 & 22.80 & 156.91 & 699.00 \\
 6 & (10, 10, 30) & 897.86 & 15.00 & 160.87 & 722.00 \\
 7 & (10, 10, 30) & 829.84 & 18.00 & 121.84 & 690.00 \\
 8 & (10, 10, 30) & 832.88 & 23.85 & 133.03 & 676.00 \\
 9 & (10, 5, 34) & 576.74 & 22.90 & -147.16 & 701.00 \\
 10 & (10, 10, 30) & 858.47 & 17.90 & 146.57 & 694.00 \\
\end{longtable}

最后得到的所有用户总体最大用户服务质量$Y_{total}$=7390.90 

\textbf{2.分析}
从表格数据来看,该方案在10个决策周期内,采用动态调整资源分配策略,成功应对了用户和网络环境的变化,实现了总服务质量(QoS)的累积最大化。具体分析如下:

\begin{itemize}
 \item \textbf{资源分配策略的动态性}:
 决策轮数1-4和第9轮采用了 (1, 1, 17) 的资源分配方案(即为URLLC、eMBB、mMTC分配的资源块数量)。决策轮数5-8和第10轮采用了 (1, 2, 15) 的方案。结果显示,当采用 (1, 2, 15) 的方案时,总 QoS 值通常显著高于 (1, 1, 17) 方案,尤其是在第5、6、10轮,总 QoS 超过了850,这表明 (1, 2, 15) 方案在某些动态条件下更为有效。

 \item \textbf{各切片服务质量的贡献}:
 $QoS_{mMTC}$ 是总服务质量的最大贡献者,其数值一直保持在600以上,这与mMTC切片侧重于大规模连接性、效用值累加的特性相符。$QoS_{URLLC}$ 的数值相对稳定,波动较小,表明对URLLC用户的服务质量能保持在一个相对可靠的水平。$QoS_{eMBB}$ 的波动性最大,不仅数值变化剧烈,甚至在第2、3、4、9轮出现了负值,这表明在这些决策周期中,eMBB用户的服务水平协议(SLA),很可能是速率或时延,未能得到满足,导致了惩罚。

 \item \textbf{动态决策的有效性}:
 尽管eMBB的QoS波动很大,但方案在第5轮和第10轮成功切换了资源分配,使总QoS从较低水平大幅提升,并有效地改善了eMBB的QoS,使其从负值转为正值,甚至达了150以上的较高水平。这证明了模型能根据动态变化的用户任务和信道条件,调整资源分配策略,以避免惩罚并提升整体服务质量。
\end{itemize}


\subsection{问题三模型建立与求解}

\subsubsection{模型建立}
本问题可建模为一个多阶段动态联合优化问题,其目标函数和约束条件如下:

\textbf{1.1 目标函数}
目标是最大化所有决策周期 $t$ 内所有用户总体服务质量:
$$\max \quad Y_{total} = \sum_{t=1}^{10} \sum_{k \in U(t) \cup E(t) \cup M(t)} y_k(t)$$
其中,$y_k(t)$ 是用户 $k$ 在决策周期 $t$ 的服务质量。

\textbf{1.2 决策变量}
决策变量包括每个基站的功率分配和资源块分配,且两者都在一定范围内变化:
\begin{itemize}
 \item 发射功率:$p_j(t)$,表示基站 $j$ 在决策周期 $t$ 的发射功率,其取值范围为 [10, 30] dBm。
 \item 资源块分配:$n_{\text{切片},j}(t)$,表示基站 $j$ 在决策周期 $t$ 分配给某一切片的资源块数量。
\end{itemize}

\textbf{1.3 约束条件}
\begin{enumerate}
 \item \textbf{发射功率约束}:每个基站的发射功率是离散的,范围在10dBm到30dBm之间。此外,发射功率的波动范围受到限制:
 $$p_j(t) \in [10, 30], \quad p_j(t) \in \mathbb{N}$$
 $$|p_j(t) - p_j(t-1)| \le \Delta p_{max}, \quad \forall t > 1$$
 \item \textbf{资源块总数约束}:每个基站分配的资源块不超过50。
 $$\sum_{\text{切片}} n_{\text{切片},j}(t) \le 50$$
 \item \textbf{资源块波动约束}:为提高算法效率,分配给每一切片的资源块数量波动受到限制:
 $$|n_{\text{切片},j}(t) - n_{\text{切片},j}(t-1)| \le \Delta n_{max}, \quad \forall t > 1$$
 \item \textbf{资源块单位约束}:分配给每类切片的资源块数量必须为该切片每个任务所需资源块数量的整数倍。
 $$n_{URLLC,j}(t) \in \{10k_1 | k_1 \in \mathbb{N}_{\ge 0}\} $$
 $$n_{eMBB,j}(t) \in \{5k_2 | k_2 \in \mathbb{N}_{\ge 0}\} $$
 $$n_{mMTC,j}(t) \in \{2k_3 | k_3 \in \mathbb{N}_{\ge 0}\} $$
 其中,$|U_{j}(t)|$、 $|E_{j}(t)|$ 和 $|M_{j}(t)|$ 分别表示在决策周期 $t$ 内接入j基站的有任务的 URLLC、eMBB 和 mMTC 用户数量。
 \item \textbf{用户接入约束}:用户在接入决策时,基于与基站的距离,采取以下策略:

设在100ms的决策周期内,用户 $k$ 与基站 $j$ 在各个时刻的距离集合为 $D_{k,j} = \{d_{k,j}(t')\}_{t'=1}^T$,其中 $T$ 为该周期内的时间点数量。

用户 $k$ 的接入决策变量为 $a_{k,j}$,当用户 $k$ 接入基站 $j$ 时,$a_{k,j} = 1$,否则 $a_{k,j} = 0$。

那么,用户 $k$ 的接入策略可表示为:
$$a_{k,j} = \begin{cases} 1, & \text{if } j = \arg\min_{j' \in \mathcal{N}_{SBS}} \{\text{Median}(D_{k,j'})\} \\ 0, & \text{otherwise} \end{cases}$$

其中:
\begin{itemize}
 \item $\mathcal{N}_{SBS}$ 代表所有微基站的集合。
 \item $\text{Median}(D_{k,j'})$ 表示在决策周期内,用户 $k$ 与基站 $j'$ 之间所有距离的中位数。
 \item $\arg\min_{j'}$ 表示使中位数距离最小的那个基站 $j'$。
\end{itemize}

另外,每个用户在每个决策周期只能接入一个基站,即:

$$\sum_j a_{k,j}(t) = 1, \quad \forall k, t$$
 \item \textbf{信干噪比(SINR)计算}:因存在干扰,信干噪比 $\gamma$ 的计算需考虑其他基站的干扰信号。
 $$\gamma=\frac{p_{n,k}\phi_{n,k}|h_{n,k}|^{2}}{\sum_{u\in N,u\ne n}p_{u,k}\phi_{u,k}|h_{u,k}|^{2}+N_{0}}$$
 其中,分母中的干扰项 $\sum_{u\in N,u\ne n}p_{u,k}\phi_{u,k}|h_{u,k}|^{2}$ 不再为0。
\end{enumerate}

\subsubsection{求解方法}
本问题是一个复杂的混合整数非线性规划问题。考虑到任务到达分布(泊松分布、均匀分布)是无记忆的,及功率是离散值,可采用穷举、贪心与束搜索相结合的启发式算法求解。

\textbf{求解思路}
\begin{enumerate}
 \item \textbf{功率调优(外层问题)}:将发射功率 $p_j(t)$ 视为超参数调优。在第一个决策周期 ($t=1$),对所有基站的功率在 [10, 30] dBm 范围内完全遍历以寻找最优初始解。在后续决策周期 ($t > 1$),则基于前一周期的最优功率解,在 $\Delta p_{max}$ 范围内有限搜索。
 \item \textbf{资源分配(内层问题)}:对每个固定的功率值,资源分配问题成为一个动态规划问题。在第一个决策周期,对资源分配方案采用贪心策略,而在后续周期 ($t > 1$),则采用束搜索算法在 $\Delta n_{max}$ 范围内搜索,以应对用户任务的动态到达和移动性。
 \item \textbf{用户接入策略}:用户在每个决策周期内根据与基站的距离接入决策。每个用户 $k$ 会选择与其距离最小的基站 $j$ 接入,确保其接入的基站能提供最佳的服务质量。
 \item \textbf{最优解选取}:在遍历有限的功率组合和资源块分配方案后,结合用户接入策略,计算每种组合下所有用户的总QoS,选取能使总服务质量达最大的方案作为最优解。
\end{enumerate}

\subsubsection{结果分析}
\begin{longtable}{cccccc}
 \caption{每一次的资源块分配与功率决策} \\
 \toprule
 \textbf{决策轮数 (t)} & \textbf{BS1 分配} & \textbf{BS2 分配} & \textbf{BS3 分配} & \textbf{功率 (dBm)} \\
 \midrule
 \endfirsthead

 \multicolumn{5}{c}%
 {\tablename\ \thetable\ -- 续表} \\
 \toprule
 \textbf{决策轮数 (t)} & \textbf{BS1 分配} & \textbf{BS2 分配} & \textbf{BS3 分配} & \textbf{功率 (dBm)} \\
 \midrule
 \endhead

 \bottomrule
 \endlastfoot
 1 & (10, 5, 34) & (10, 5, 34) & (10, 5, 34) & (27, 29, 30) \\
 2 & (10, 5, 34) & (10, 5, 34) & (10, 5, 34) & (24, 30, 29) \\
 3 & (10, 5, 34) & (10, 5, 34) & (10, 5, 34) & (25, 27, 30) \\
 4 & (10, 5, 34) & (10, 5, 34) & (10, 5, 34) & (28, 30, 30) \\
 5 & (10, 5, 34) & (10, 5, 34) & (10, 5, 34) & (25, 30, 28) \\
 6 & (10, 5, 34) & (10, 5, 34) & (10, 5, 34) & (23, 30, 26) \\
 7 & (10, 5, 34) & (10, 5, 34) & (10, 5, 34) & (21, 30, 23) \\
 8 & (10, 5, 34) & (10, 5, 34) & (10, 5, 34) & (18, 30, 25) \\
 9 & (10, 5, 34) & (10, 5, 34) & (10, 5, 34) & (20, 30, 27) \\
 10 & (10, 5, 34) & (10, 5, 34) & (10, 5, 34) & (17, 30, 24) \\
\end{longtable}

\begin{longtable}{cccccc}
 \caption{每一次的用户服务质量} \\
 \toprule
 \textbf{决策轮数 (t)} & \textbf{$Y_{total}(t)$} & \textbf{$Y_{URLLC}(t)$} & \textbf{$Y_{eMBB}(t)$} & \textbf{$Y_{mMTC}(t)$} \\
 \midrule
 \endfirsthead

 \multicolumn{5}{c}%
 {\tablename\ \thetable\ -- 续表} \\
 \toprule
 \textbf{决策轮数 (t)} & \textbf{$Y_{total}(t)$} & \textbf{$Y_{URLLC}(t)$} & \textbf{$Y_{eMBB}(t)$} & \textbf{$Y_{mMTC}(t)$} \\
 \midrule
 \endhead

 \bottomrule
 \endlastfoot
 1 & 1987.865 & 26.949 & 1.916 & 1959.000 \\
 2 & 2187.840 & 46.398 & 1.443 & 2140.000 \\
 3 & 2172.110 & 35.553 & -5.443 & 2142.000 \\
 4 & 2143.959 & 22.243 & -5.285 & 2127.000 \\
 5 & 2163.985 & 37.799 & -1.815 & 2128.000 \\
 6 & 2201.635 & 48.552 & -4.917 & 2158.000 \\
 7 & 2178.306 & 32.052 & 2.254 & 2144.000 \\
 8 & 2189.852 & 46.286 & -7.434 & 2151.000 \\
 9 & 2148.680 & 54.806 & -7.127 & 2101.000 \\
 10 & 2233.007 & 49.524 & -3.517 & 2187.000 \\
\end{longtable}

本模型在处理多微基站、多切片、存在干扰的复杂无线资源管理问题时,表现出了显著的优秀性。其核心优势在于对资源块分配、功率控制和用户接入策略的综合考量与优化。



\textbf{1. 资源分配的稳定性与效用最大化}
该模型在10个决策周期内,为每个基站分配的资源块数量保持不变,分别为URLLC切片10个,eMBB切片5个,及mMTC切片34个。每个基站的总资源块数量为49,符合每个基站不超过50个资源块的约束。这种稳定的分配策略并非简单的静态分配,而是模型在面对动态任务到达和信道变化时,经过联合优化后找到的全局最优或次优解。它反映了以下几点:
\begin{itemize}
 \item \textbf{对业务需求的深刻理解}:URLLC切片以极低时延和高可靠性为主要特点,分配10个资源块可确保其任务快速完成,从而满足5ms的时延SLA要求。mMTC切片则专注于支持大规模设备连接,分配34个资源块以确保高接入比例,满足其SLA要求。eMBB切片虽然对速率要求高(50Mbps),但在资源有限的情况下需平衡其他切片的需求,模型为其分配了5个资源块。这种分配方式体现了模型对不同切片SLA的优先级排序和权衡。
 \item \textbf{求解效率的提升}:固定的资源块分配方案可显著降低后续决策周期的计算复杂度。模型可能采用对用户任务到达分布(URLLC为泊松分布,eMBB和mMTC为均匀分布)和信道条件的长期统计分析,找到了一个在1000ms内都有鲁棒性的资源配置方案。
\end{itemize}

\textbf{2. 动态功率控制与干扰抑制}
与资源分配的静态性形成对比,模型的功率控制部分展现出极强的动态自适应能力。
\begin{itemize}
 \item \textbf{有效应对基站间干扰}:在多微基站场景下,不同基站因频率复用会产生相互干扰。信干噪比(SINR)的计算明确包含了来自其他基站的干扰项 $\sum_{u\in N,u\ne n}p_{u,k}\phi_{u,k}|h_{u,k}|^{2}$。该模型采用在每个决策周期动态调整每个基站的发射功率(取值范围为10-30dBm),有效地控制了干扰,从而保证了用户能获得更高的SINR,进而提升了传输速率和用户服务质量。
 \item \textbf{平衡QoS与能耗}:尽管该分析未直接涉及能耗,但动态功率调整为后续能耗优化提供了基础。模型在保证总服务质量最大化的同时,采用适时降低发射功率,可减少能耗,为运营商带来经济效益。这体现了模型在多目标优化方面的潜力。
\end{itemize}

\textbf{3. 综合评估与联合优化}
该模型最核心的优势在于其联合优化了资源块分配和功率控制,以最大化所有用户的总服务质量。
\begin{itemize}
 \item \textbf{全局最优视角}:模型的目标函数是最大化所有决策周期内所有用户的总服务质量。这表明模型不是简单地优化单个用户的性能,而是从系统全局角度出发,寻求一个能让所有用户都获得最佳服务的均衡点。
 \item \textbf{对SLA惩罚的应对}:eMBB切片服务质量出现负值的情况,如决策周期3、4、5等,这反映了模型在某些时段为优先满足URLLC和mMTC用户的SLA,会牺牲部分eMBB用户的QoS。然而,因URLLC和mMTC用户的QoS保持在较高水平,总服务质量依然很高。这表明模型能权衡不同切片的SLA,并根据惩罚系数$M$(URLLC为5,eMBB为3,mMTC为1)优先级排序,以确保整体效用最大化。
\end{itemize}

综上所述,该模型采用分层的启发式求解方法,有效地解决了多基站、多切片、动态任务和信道变化的复杂资源管理问题。其稳定的资源块分配方案和动态的功率控制机制相辅相成,共同实现了对整体用户服务质量的最大化,是该问题一个优秀且可行的解决方案。






\subsection{问题四模型建立与求解}
\subsubsection{模型建立}
问题四将网络场景扩展至包含宏基站(MBS)和微基站(SBS)的异构网络,要求对用户接入、资源分配和功率控制联合优化,以最大化系统的总服务质量(QoS)。

\textbf{1.1 目标函数}
目标是最大化所有决策周期 $t$ 内所有用户的总服务质量:
$$\max \quad Y_{total} = \sum_{t=1}^{10} \sum_{k \in U(t) \cup E(t) \cup M(t)} y_k(t)$$
其中,$y_k(t)$ 是用户 $k$ 在决策周期 $t$ 的服务质量。

\textbf{1.2 决策变量}
决策变量包括用户接入决策、每个基站的功率分配和资源块分配:
\begin{itemize}
 \item 用户接入决策:$a_{k,j}(t)$,一个二元变量,当用户 $k$ 在决策周期 $t$ 接入基站 $j$ 时,$a_{k,j}(t)=1$,否则为0。
 \item 发射功率:$p_j(t)$,表示基站 $j$ 在决策周期 $t$ 的发射功率。
 \item 资源块分配:$n_{\text{切片},j}(t)$,表示基站 $j$ 在决策周期 $t$ 分配给某一切片的资源块数量。
\end{itemize}

\textbf{1.3 约束条件}
\begin{enumerate}
 \item \textbf{发射功率约束}:MBS和SBSs的功率范围不同。MBS的功率范围为 [10, 40] dBm,SBSs的功率范围为 [10, 30] dBm。
 \item \textbf{资源块总数约束}:MBS和SBSs的资源块数量不同。MBS有100个资源块,SBSs有50个资源块。
 \item \textbf{用户接入约束}:每个用户在每个决策周期只能接入一个基站。
 $$\sum_j a_{k,j}(t) = 1, \quad \forall k, t$$
 因为加入了一个新的宏基站,$a_{k,j}$的新表达式为:

 $$a_{k,j} = \begin{cases} 1, & \text{if } j = \arg\min_{j' \in \mathcal{N}_{SBS} \cup \mathcal{N}_{MBS}} \{\text{Median}(D_{k,j'})\} \\ 0, & \text{otherwise} \end{cases}$$

其中:
\begin{itemize}
 \item $\mathcal{N}_{SBS}$,$\mathcal{N}_{MBS}$ 代表所有微基站和宏基站的集合。
 \item $\text{Median}(D_{k,j'})$ 表示在决策周期内,用户 $k$ 与基站 $j'$ 之间所有距离的中位数。
 \item $\arg\min_{j'}$ 表示使中位数距离最小的那个基站 $j'$。
\end{itemize}
 \item \textbf{干扰模型与信干噪比(SINR)计算}:MBS和SBS的频谱不重叠,彼此间不存在干扰。但SBSs之间存在干扰。因此,用户信干噪比 $\gamma$ 的计算需根据其接入的基站类型区分。
 \begin{itemize}
 \item 白噪声 $N_0$:$N_{0}=-174+10\log_{10}(ib)+NF$。
 \item MBS用户:SINR计算中不包含干扰项。
 \item SBS用户:SINR计算需包含来自其他SBSs的干扰项。
 $$\gamma=\frac{p_{n,k}\phi_{n,k}|h_{n,k}|^{2}}{\sum_{u\in N,u\ne n}p_{u,k}\phi_{u,k}|h_{u,k}|^{2}+N_{0}}$$
 其中,分母中的干扰项 $\sum_{u\in N,u\ne n}p_{u,k}\phi_{u,k}|h_{u,k}|^{2}$ 仅对SBS用户有效。
 \end{itemize}
\end{enumerate}

\subsubsection{求解方法}
问题四是一个复杂的混合整数非线性规划问题,涉及用户接入、资源分配和功率控制三类耦合决策。为解决这一难题,本文采用一种分层迭代的启发式算法。

\textbf{求解思路}
\begin{enumerate}
 \item \textbf{初始化}:基于用户与基站的初始距离或信道信息,为所有用户分配一个初始接入方案。
 \item \textbf{迭代优化}:
 \begin{itemize}
 \item \textbf{基站内优化}:在当前的用户接入方案下,每个基站(MBS和SBS)独立求解其内部的资源分配和功率控制问题。
 
 * MBS:其优化问题相对简化,主要目标是在100个资源块和 [10, 40] dBm 的功率范围内最大化其接入用户的 QoS。
 
 * SBSs:其优化问题可沿用问题三的解耦策略:将功率视为超参数,在 [10, 30] dBm 范围内调优,同时采用束搜索算法解决资源块的动态分配,以实现最大的用户Qos。
 
 \item \textbf{用户接入更新}:每次决策后采用位置大小关系对用户的接入方案更新。具体而言,用户根据与各基站的距离重新选择接入的基站,以确保其接入的基站能提供最佳的服务质量。
 \end{itemize}
 \item \textbf{收敛判断}:重复上述迭代过程,在遍历有限的功率组合和资源块分配方案后,结合用户接入策略,计算每种组合下所有用户的总QoS,选取能使总服务质量达最大的方案作为最优解。
\end{enumerate}

\subsubsection{结果分析}
\begin{longtable}{cccccc}
 \caption{每一次的资源块分配与功率决策} \\
 \toprule
 \textbf{决策轮数 (t)} & \textbf{BS1 分配} & \textbf{BS2 分配} & \textbf{BS3 分配} & \textbf{MBS 分配} & \textbf{功率 (dBm)} \\
 \midrule
 \endfirsthead

 \multicolumn{6}{c}%
 {\tablename\ \thetable\ -- 续表} \\
 \toprule
 \textbf{决策轮数 (t)} & \textbf{BS1 分配} & \textbf{BS2 分配} & \textbf{BS3 分配} & \textbf{MBS 分配} & \textbf{功率 (dBm)} \\
 \midrule
 \endhead

 \bottomrule
 \endlastfoot
 1 & (10, 5, 34) & (20, 5,24) & (10, 10, 30) & (10, 35, 44) & (30,30,29,40) \\

 2 & (10, 10, 30) & (20, 5,24) & (20, 5, 24) & (20, 55, 24) & (30,30,27,39) \\

 3 & (10, 10, 30) & (10, 10, 30) & (20, 5, 24) & (20, 40, 40) & (30,28,25,37) \\

 4 & (10, 5, 34) & (20, 5, 24) & (20, 10, 20) & (10, 30, 60) & (30,26,23,38) \\

 5 & (10, 5, 34) & (20, 10, 20) & (20, 10, 20) & (10, 50, 40) & (30,24,21,38) \\

 6 & (10, 5, 34) & (20, 10, 20) & (10, 10, 30) & (20, 40, 40) & (28,30,25,39) \\

 7 & (10, 5, 34) & (20, 10, 20) & (20, 10, 20) & (10, 60, 30) & (30,23,17,40) \\

 8 & (10, 5, 34) & (20, 10, 20) & (20, 10, 20) & (20, 60, 20) & (30,23,15,40) \\

 9 & (10, 5, 34) & (20, 10, 20) & (10, 15, 24) & (10, 35, 54) & (30,23,16,40) \\

 10 & (10, 5, 34) & (20, 10, 20) & (20, 10, 20) & (40, 35, 24) & (30,24,14,40) \\
\end{longtable}

\begin{longtable}{cccccc}
 \caption{每一次的用户服务质量} \\
 \toprule
 \textbf{决策轮数 (t)} & \textbf{$Y_{total}(t)$} & \textbf{$Y_{URLLC}(t)$} & \textbf{$Y_{eMBB}(t)$} & \textbf{$Y_{mMTC}(t)$} \\
 \midrule
 \endfirsthead

 \multicolumn{6}{c}%
 {\tablename\ \thetable\ -- 续表} \\
 \toprule
 \textbf{决策轮数 (t)} & \textbf{$Y_{total}(t)$} & \textbf{$Y_{URLLC}(t)$} & \textbf{$Y_{eMBB}(t)$} & \textbf{$Y_{mMTC}(t)$} \\
 \midrule
 \endhead

 \bottomrule
 \endlastfoot

 1 & 2192.624 & 59.902 & 148.722 & 1984.000 \\
 2 & 2332.129 & 61.763 & 153.366 & 2117.000 \\
 3 & 2351.819 & 54.767 & 152.051 & 2145.000 \\
 4 & 2507.876 & 59.651 & 142.225 & 2306.000 \\
 5 & 2506.382 & 52.810 & 160.572 & 2293.000 \\
 6 & 2460.150 & 45.206 & 140.944 & 2274.000 \\
 7 & 2447.993 & 44.034 & 150.959 & 2253.000 \\
 8 & 2429.281 & 52.096 & 146.185 & 2231.000 \\
 9 & 2387.429 & 39.007 & 135.421 & 2213.000 \\
 10 & 2401.121 & 29.780 & 134.341 & 2237.000 \\
\end{longtable}

根据上表所得结果,该模型成功地解决了在宏微混合异构网络中的联合优化问题,在10个决策周期内,采用动态调整资源分配和功率控制,实现了高水平且稳定的总服务质量。

\textbf{总服务质量(QoS)分析:}
\begin{itemize}
 \item 总QoS值持续保持在2192以上,最高达了2507.876(第4轮),这表明异构网络和动态优化策略的结合能显著提升系统性能。
 \item $QoS_{mMTC}$ 仍然是总QoS的主要贡献者,其数值始终在1900以上,这与大规模物联网通信任务量大的特性相符。
\end{itemize}

\textbf{切片间的权衡策略:}
\begin{itemize}
 \item 与之前的结果不同,本次运行中所有决策轮次的 $QoS_{URLLC}$ 和 $QoS_{eMBB}$ 值都为正,这表明模型在满足URLLC和eMBB的SLA时,仍能获得可观的mMTC QoS。这体现了模型在新的动态场景下,找到了一个更好的平衡点,能有效避免惩罚,同时最大化总服务质量。
\end{itemize}

\textbf{异构网络中的基站角色:}
\begin{itemize}
 \item \textbf{宏基站(MBS)}的发射功率(35-40 dBm)远高于微基站,且其资源分配方案的调整更为灵活(如第1轮的(2, 7, 22),第5轮的(1, 10, 20)),这符合其作为网络骨干、提供广域覆盖和动态资源调度的角色定位。
 \item \textbf{微基站(SBS)}的功率和资源分配方案也随时间波动,以适应局部用户的需求并管理相互间的干扰。如,在第1轮,BS2的分配方案是(2, 1, 12),但在第2轮就调整为(2, 1, 12),这表明模型在每个决策周期都在积极地优化以适应环境变化。
\end{itemize}
该结果充分展示了模型在复杂异构网络中,能采用动态的功率和资源分配策略,在多目标之间做出智能权衡,从而实现系统性能的最大化。

\subsection{问题五模型建立与求解}

\subsubsection{模型建立}
问题五是一个多目标优化问题,旨在最大化用户的总服务质量,同时最小化系统的能耗。该问题可建模为一个多阶段动态规划模型,其目标函数和约束条件如下:

\textbf{1.1 目标函数}
目标是最大化所有决策周期 $t$ 内所有用户的总服务质量,同时最小化系统的能耗:
$$\max \quad F= Y_{total}-\mu P_{total}$$

其中:$\mu $为超参数,是为平衡基站能耗对用户服务质量的影响。$Y_{total}$ 是所有用户在所有决策周期内的总服务质量,$P_{total}$ 是系统在所有决策周期内的总能耗。

\textbf{1.2 决策变量}
决策变量包括用户接入决策、每个基站的功率分配和资源块分配:
\begin{itemize}
 \item 用户接入决策:$a_{k,j}(t)$,一个二元变量,当用户 $k$ 在决策周期 $t$ 接入基站 $j$ 时,$a_{k,j}(t)=1$,否则为0。
 \item 发射功率:$p_j(t)$,表示基站 $j$ 在决策周期 $t$ 的发射功率。
 \item 资源块分配:$n_{\text{切片},j}(t)$,表示基站 $j$ 在决策周期 $t$ 分配给某一切片的资源块数量。
\end{itemize}

\textbf{1.3 约束条件}
\begin{enumerate}
 \item \textbf{发射功率约束}:MBS的功率范围为 [10, 40] dBm,SBSs的功率范围为 [10, 30] dBm。
 \item \textbf{资源块总数约束}:MBS有100个资源块,SBSs有50个资源块。
 \item \textbf{用户接入约束}:每个用户在每个决策周期只能接入一个基站。
 $$\sum_j a_{k,j}(t) = 1, \quad \forall k, t$$
 \item \textbf{干扰模型与信干噪比(SINR)计算}:MBS和SBS的频谱不重叠,彼此间不存在干扰。但SBSs之间存在干扰。
 $$\gamma=\frac{p_{n,k}\phi_{n,k}|h_{n,k}|^{2}}{\sum_{u\in N,u\ne n}p_{u,k}\phi_{u,k}|h_{u,k}|^{2}+N_{0}}$$
 其中,分母中的干扰项 $\sum_{u\in N,u\ne n}p_{u,k}\phi_{u,k}|h_{u,k}|^{2}$ 仅对SBS用户有效。
 \item \textbf{能耗计算模型}:系统总能耗 $P_{total}$ 为所有基站在所有决策周期内能耗的累加。单个基站 $j$ 在决策周期 $t$ 的总能耗 $P_j(t)$ 由三部分组成,分别为固定能耗、RB激活能耗及发射能耗:
 \begin{itemize}
 \item 固定能耗:$P_{static}=28~W$。
 \item RB激活能耗:$P_{RB}=\delta\times N_{active}$,其中 $\delta=0.75(W/RB)$。 $N_{active}$ 为基站 $j$ 在决策周期 $t$ 内激活的资源块数量。
 \item 发射功耗:$P_{tx}=\frac{1}{\eta}p_{transmit}$,其中 $\eta=0.35$。
 \end{itemize}
 值得注意的是,在计算发射功耗时,需将发射功率单位 dBm 转换为 W,其公式为:$p_{(W)}=10^{\frac{p_{(dBm)}-30}{10}}$。
\end{enumerate}

\subsubsection{求解方法}
本问题是一个多目标优化问题,旨在最大化用户服务质量并最小化能耗。为解决这一难题,本文采用罚函数法将其转化为一个单目标优化问题,并利用启发式搜索算法求解。

\textbf{1.参数化分析与初始化}

在多目标优化问题中,$\mu$ 是一个超参数,用于平衡相互冲突的目标。它的值代表了两个目标函数之间的\textbf{边际替代率},即\textbf{帕累托前沿}的斜率。因此,$\mu$ 并非一个固定值,而是需根据对 QoS 和能耗的权衡偏好来设定的。

为确定一个合适的 $\mu$ 值,并探索 QoS 与能耗之间的最优权衡,本文采用\textbf{参数化分析}的方法:

\begin{enumerate}
 \item \textbf{参数化求解}:选择一个 $\mu $ 的取值范围,并用多个离散值对这个范围采样。
 \item \textbf{求解模型}:对每个选定的 $\mu$ 值,求解问题五中的单目标优化模型,从而得到一个对应的最优解,该解包含一组功率分配、资源分配方案及由此产生的总服务质量 $Y_{total}$ 和总能耗 $P_{total}$。
 \item \textbf{帕累托前沿分析}:将每一组 ($P_{total}$, $Y_{total}$) 绘制在图上,由此形成的曲线即为帕累托前沿。
 \item \textbf{选择最优方案}:采用分析帕累托前沿,可根据实际需求,选择一个能满足服务质量要求,又能将能耗控制在合理范围内的方案。

 $\mu$ 的值可大致被看作是 $\frac{\Delta Y_{total}}{\Delta P_{total}}$,它代表了为换取一个单位的服务质量,所需付出的能耗代价。因此,采用调整 $\mu$ 的值,本文可在帕累托前沿上找到不同的平衡点,从而获得一系列最优的权衡方案。

采用真实的图像分析,当$P_{total}$ 处于一个较低水平时,$\mu \approx 1$;当$P_{total}$ 处于一个较高水平时,$\mu \approx 44$.本文取一个介于其间的值 $\mu = 20$ 后续的求解。
 \item \textbf{取值合理性}:$\mu =20$ 是一个介于低功率区域($\mu ≈1$)和高功率区域($\mu ≈44$)之间的值。这意味着本文所找到的解位于帕累托前沿的中间地带,是一个相对高效的权衡点,能避免在能耗上投入过多但回报(QoS)却递减的情况。
 
 在实际应用中,运营商可能需一个能提供较好用户体验(高QoS)同时又能有效控制运营成本(低能耗)的方案。选择 $\mu =20$ 恰好能体现这种既要保证服务质量,又要兼顾能效的策略,是一个有现实意义的决策。
\end{enumerate}
\textbf{2. 求解算法}
在确定 $\mu$ 值后,问题转化为一个单目标优化问题,可沿用问题四的分层迭代算法求解,但评估指标变为新的综合目标函数 $F$。
\begin{enumerate}
 \item \textbf{初始化}:基于用户与基站的初始距离或信道信息,为所有用户分配一个初始接入方案。
 \item \textbf{迭代优化}:
 \begin{itemize}
 \item \textbf{基站内优化}:在当前的用户接入方案下,每个基站(MBS和SBS)独立求解其内部的资源分配和功率控制问题。优化目标是最大化综合效用函数 $F$。
 \item \textbf{用户接入更新}:每次决策后采用贪心算法对用户的接入方案更新。具体而言,用户根据与各基站的距离重新选择接入的基站,以确保其接入的基站能提供最佳的服务质量。
 \end{itemize}
 \item \textbf{收敛判断}:重复上述迭代过程,直到用户接入方案不再发生变化,或系统的综合效用值 $F$ 提升微乎其微,即可认为找到一个稳定的近似最优解。
\end{enumerate}

\subsubsection{结果分析}
\begin{longtable}{cccccc}
 \caption{每一次的资源块分配与功率决策} \\
 \toprule
 \textbf{决策轮数 (t)} & \textbf{BS1 分配} & \textbf{BS2 分配} & \textbf{BS3 分配} & \textbf{MBS 分配} & \textbf{功率 (dBm)} \\
 \midrule
 \endfirsthead

 \multicolumn{6}{c}%
 {\tablename\ \thetable\ -- 续表} \\
 \toprule
 \textbf{决策轮数 (t)} & \textbf{BS1 分配} & \textbf{BS2 分配} & \textbf{BS3 分配} & \textbf{MBS 分配} & \textbf{功率 (dBm)} \\
 \midrule
 \endhead

 \bottomrule
 \endlastfoot

 1 & (10, 5, 34) & (10, 5, 34) & (10, 5, 34) & (10, 30, 60) & (26,30,26,40) \\
 2 &(10, 5, 34) & (10, 5, 34) & (10, 5, 34) & (10, 50, 40) & (27,30,23,39) \\
 3 & (10, 5, 34) &(10, 5, 34) & (10, 5, 34) & (20, 40, 40) & (30,27,23,35) \\
 4 & (10, 5, 34) & (10, 5, 34) & (10, 5, 34) & (10, 30, 60) & (30,30,22,39) \\
 5 & (10, 5, 34) & (10, 5, 34) & (10, 5, 34) & (10, 50, 40) & (30,28,22,38) \\
 6 & (10, 5, 34) & (10, 5, 34) & (10, 10, 30) & (20, 40, 40) & (28,30,25,39) \\
 7 & (10, 5, 34) & (10, 5, 34)& (10, 5, 34) & (10, 50, 40) & (30,27,21,40) \\
 8 & (10, 5, 34) & (10, 5, 34) & (10, 5, 34) & (20, 35, 44) & (30,23,18,40) \\
 9 & (10, 5, 34) & (10, 10, 30) & (10, 15, 24) & (10, 35, 54) & (30,20,21,40) \\
 10 & (10, 5, 34) & (10, 10, 30) & (10, 5, 34) & (10, 40, 50) & (30,23,17,40) \\
\end{longtable}

\begin{longtable}{cccccc}
 \caption{每一次的用户服务质量} \\
 \toprule
 \textbf{决策轮数 (t)} & \textbf{$Y_{total}(t)$} & \textbf{$Y_{URLLC}(t)$} & \textbf{$Y_{eMBB}(t)$} & \textbf{$Y_{mMTC}(t)$} \\
 \midrule
 \endfirsthead

 \multicolumn{6}{c}%
 {\tablename\ \thetable\ -- 续表} \\
 \toprule
 \textbf{决策轮数 (t)} & \textbf{$Y_{total}(t)$} & \textbf{$Y_{URLLC}(t)$} & \textbf{$Y_{eMBB}(t)$} & \textbf{$Y_{mMTC}(t)$} \\
 \midrule
 \endhead

 \bottomrule
 \endlastfoot

 1 & 2192.672 & 47.134 & 145.538 & 2000.000 \\
 2 & 2342.284 & 53.307 & 149.977 & 2139.000 \\
 3 & 2304.460 & 54.403 & 151.056 & 2099.000 \\
 4 & 2323.526 & 51.420 & 136.106 & 2136.000 \\
 5 & 2305.440 & 22.126 & 154.315 & 2129.000 \\
 6 & 2264.845 & 56.397 & 144.448 & 2064.000 \\
 7 & 2340.813 & 37.188 & 150.625 & 2153.000 \\
 8 & 2334.538 & 26.573 & 139.965 & 2168.000 \\
 9 & 2326.991 & -21.584 & 141.574 & 2207.000 \\
 10 & 2215.380 & -18.928 & 134.308 & 2100.000 \\
\end{longtable}

\textbf{资源块分配与功率决策}
\begin{itemize}
 \item \textbf{资源分配:} 从表格中可看出,MBS 分配的资源块数量(100个)远多于 SBSs(50个),因此 MBS 在大多数决策周期中承担了更多的资源分配任务。值得注意的是,MBS 的资源分配在不同决策周期波动较大,如在 $t=1$ 时为 (1, 6, 30),而在 $t=9$ 时变为 (1, 7, 27)。这可能反映了 M-MTC 用户(第三个分量)数量的变化或其对资源需求的动态调整。
 \item \textbf{功率决策:} 功率决策也展现出动态调整的特点。MBS 的发射功率始终处于较高的水平(35-40 dBm),这与其广覆盖范围和支持大量用户的角色相符。而 SBSs 的功率则在 $[17, 30]$ dBm 之间波动,这可能与它们所服务的用户数量、用户位置及网络中其他 SBSs 的干扰情况有关。这种动态的功率调整是算法为平衡 QoS 和能耗而做出的决策。
\end{itemize}

\textbf{用户服务质量}
\begin{itemize}
 \item \textbf{$Y_{\text{total}}(t)$:} 系统的总服务质量在大部分决策周期内保持在 2200-2350 之间,但在 $t=10$ 时下降至 2215.380。这可能表明在 $t=10$ 时,网络做出了更倾向于节能的决策,牺牲了一部分 QoS。
 \item \textbf{$Y_{\text{URLLC}}(t)$:} URLLC 服务的 QoS 在大部分时间都保持在正值,但在 $t=9$ 和 $t=10$ 时变为负值,分别为 -21.584 和 -18.928。这需引起特别关注。在实际应用中,URLLC 服务对时延和可靠性要求极高,其 QoS 通常不能为负。这可能意味着在某些决策周期,系统未能为 URLLC 用户提供足够的资源或功率,导致其服务质量严重不达标。这个现象可能揭示了模型在 URLLC 需求较高的场景下,可能需更严格的约束或更优先的资源调度策略。
 \item \textbf{$Y_{\text{eMBB}}(t)$ 和 $Y_{\text{mMTC}}(t)$:} eMBB 和 mMTC 的服务质量相对稳定,特别是在 mMTC 中,大部分周期都接近或超过 2000。这表明该算法在处理对带宽和连接密度要求较高的业务类型时表现良好。
\end{itemize}

\textbf{综合结论}
总体而言,该模型和求解方法在平衡 QoS 和能耗方面是有效的。然而,结果分析揭示了在 URLLC 服务的 QoS 保证方面存在潜在问题。未来的优化方向可是在模型中引入对 URLLC 服务质量的硬性约束,如确保其 QoS 始终为正值,以确保关键业务的可靠性。






\section{模型评估与敏感度分析}
本模块对问题二中建立的束搜索模型评估和敏感度分析,重点考察启发式函数的设计及各种常量对最终求解结果的影响。
\subsection{模型评估}

$$H = \text{QoS} + \begin{cases} \frac{\lambda}{m} & m \ne 0 \\ {\varepsilon} & m = 0 \end{cases}$$
 其中:
 \begin{itemize}
 
 \item $\lambda$ 是一个超参数,用于平衡当前 QoS 和未来积压任务量的重要性,这里设为50。$\lambda$ 的值越大,算法在选择方案时越倾向于减少未来积压任务,即使这可能牺牲部分当前 QoS。
 \item $\varepsilon$ 是一个常量,这里设为150。
 \end{itemize}
若本文将启发函数从原问题二求解中只计算Qos的形式改为添加积压任务量影响的方式,得到的结果如下表所示:

\begin{longtable}{cccccc}
 \caption{每一次的资源块分配决策及对应的服务质量} \\
 \toprule
 \textbf{决策轮数(t)} & \textbf{分配决策} & \textbf{$Y_{total}(t)$} & \textbf{$Y_{URLLC}(t)$} & \textbf{$Y_{eMBB}(t)$} & \textbf{$Y_{mMTC}(t)$} \\
 \midrule
 \endfirsthead

 \multicolumn{6}{c}%
 {\tablename\ \thetable\ -- 续表} \\
 \toprule
 \textbf{决策轮数(t)} & \textbf{分配决策} & \textbf{$Y_{total}(t)$} & \textbf{$Y_{URLLC}(t)$} & \textbf{$Y_{eMBB}(t)$} & \textbf{$Y_{mMTC}(t)$} \\
 \midrule
 \endhead

 \bottomrule
 \endlastfoot
 
 1 & (10, 5, 34) & 761.57 & 23.95 & 29.62 & 708.00 \\
 2 & (10, 5, 34) & 638.00 & 22.85 & -85.85 & 701.00 \\
 3 & (10, 5, 34) & 572.73 & 24.85 & -155.12 & 703.00 \\
 4 & (10, 5, 34) & 544.09 & 24.00 & -195.91 & 716.00 \\
 5 & (10, 10, 30) & 878.71 & 22.80 & 156.91 & 699.00 \\
 6 & (10, 10, 30) & 897.86 & 15.00 & 160.87 & 722.00 \\
 7 & (10, 10, 30) & 829.84 & 18.00 & 121.84 & 690.00 \\
 8 & (10, 10, 30) & 832.88 & 23.85 & 133.03 & 676.00 \\
 9 & (10, 5, 34) & 576.74 & 22.90 & -147.16 & 701.00 \\
 10 & (10, 10, 30) & 858.47 & 17.90 & 146.57 & 694.00 \\
\end{longtable}

对上表的结果,本文发现与仅考虑QoS的启发式函数相比,加入了积压任务量影响的启发式函数并不影响最后的结果,这表明问题二的建模方法和结果有极强的稳定性和鲁棒性。
\subsection{敏感度分析}
为评估模型对关键参数的敏感性,本文对以下几个常量了系统的敏感度分析:
\begin{itemize}
 \item \textbf{$\alpha$}:URLLC 用户任务效用折扣系数
 \item \textbf{任务流}:用户请求的任务量
 \item \textbf{热噪声谱密度}:$N_{0}=-174+10\log_{10}(ib)+NF$中的-174
 \item \textbf{噪声系数NF}:$N_{0}=-174+10\log_{10}(ib)+NF$中的NF,原值为7
 \item \textbf{瑞丽衰减$h_{n,k}$}:由信号在空间中多径效应造成的小规模衰减
\end{itemize} 

采用不断改变这几个常量,本文可得出它们对最后用户总体服务质量(Qos)的影响程度。结果如下图所示:
\begin{figure}[htbp]
 \centering
 \includegraphics[width=0.5\textwidth]{1.png}
 \caption{敏感度分析结果}
 \label{fig:myfigure}
\end{figure}

采用该直方图,本文可看出URLLC 用户任务效用折扣系数$\alpha$和用户请求的任务量对最终的服务质量影响最大,而热噪声谱密度、噪声系数NF和瑞丽衰减$h_{n,k}$对结果的影响则较小。这表明在实际应用中,优化用户请求的任务量和合理设置折扣系数$\alpha$是提升系统服务质量的关键因素。


\section{结论与讨论}
\subsection{研究结论}
本研究采用对五个核心问题的分析与建模,获得了明确的量化结果(详见表格\ref{}),并得出以下主要结论:
\begin{itemize}
 \item \textbf{决策收敛性:} 在问题二的求解中,研究发现当用户行为模式稳定并遵循特定概率分布时,在100ms的决策周期内,系统决策会收敛于一个特定的最优策略。
 \item \textbf{关键影响因素:} 模型分析表明,用户请求的数据流量及URLLC用户的服务质量(QoS)评价函数中的折扣因子$\alpha$是影响决策目标函数的关键变量。相比之下,其他因素(如系统噪声)的影响则不显著。
 \item \textbf{算法鲁棒性:} 采用对启发式函数变体测试,本文发现其形式对最终求解结果影响甚微。这证明了本研究采用的基于束搜索的近似贪心策略有良好的鲁棒性。
\end{itemize}

基于以上分析,本文评估了该模型在实际应用中的潜力和局限性:
\begin{itemize}
 \item \textbf{适用场景:} 该模型特别适用于用户行为可预测且稳定的环境,如工业自动化、智能楼宇等。在这些场景下,模型能在保障服务质量的同时,有效提升网络资源的利用效率。
 \item \textbf{个性化服务:} 模型充分考虑了不同用户类型在服务质量上的差异化需求,能实现资源的按需分配,从而提高用户满意度。
 \item \textbf{求解效率:} 模型对同类型用户的排队情况了精细建模,并采用束搜索算法求解,能在有限的计算资源下,高效地获得近似最优的决策方案。
 \item \textbf{局限性:} 该模型在用户行为模式发生剧烈变化的场景(如应急通信、自然灾害响应)中适用性较差,可能导致决策失效,无法满足关键时刻的用户需求。
 \item \textbf{核心假设:} 模型的有效性依赖于一个核心假设,即用户请求的数据流量在100ms决策周期内服从特定分布。在实际应用中,流量模式可能受时间、地点等多种外部因素影响,分布的动态变化可能削弱决策的有效性。
\end{itemize}

\subsection{未来工作展望}
为进一步提升模型的实用性与先进性,未来的研究可从以下方向展开:
\begin{itemize}
 \item \textbf{模型扩展:} 引入更多样化的用户类型和服务质量评价函数,以更精确地映射真实世界网络应用的多样化需求。
 \item \textbf{动态适应性:} 开发更具灵活性和自适应能力的决策算法,以应对用户行为的动态变化,特别是在突发事件和高动态场景下。
 \item \textbf{智能决策:} 融合机器学习与深度学习技术,特别是强化学习方法,以增强模型的环境感知、流量预测与自主决策能力。
 \item \textbf{实证验证:} 结合具体应用场景,开展仿真实验与物理测试,对模型的有效性、可靠性和性能全面的评估与验证。
 \item \textbf{算法优化:} 持续优化算法的计算复杂度与执行效率,以增强模型在大规模、高密度网络环境下的可扩展性与应用潜力。
\end{itemize}

\section{文献}
[1] Ji M , Chen J , Liu Z , et al. Multi-level quantization and blind equalization based direct transmission method of digital baseband signal[J]. Physical Communication, 2017:S1874490717300861.

[2] Wikipedia contributors. "Transmission time." Wikipedia, The Free Encyclopedia. Wikipedia, The Free Encyclopedia, 26 Apr. 2024. Web. 10 Aug. 2025.

[3] Chinchilla-Romero, L., Prados-Garzon, J., Ameigeiras, P., Muñoz, P., & Lopez-Soler, J. M. (2022). 5G Infrastructure Network Slicing: E2E Mean Delay Model and Effectiveness Assessment to Reduce Downtimes in Industry 4.0. Sensors, 22(1), 229. https://doi.org/10.3390/s22010229



\end{document}