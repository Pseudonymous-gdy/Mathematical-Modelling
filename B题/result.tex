\documentclass{article}
\usepackage{graphicx}
\usepackage{subcaption}
\usepackage{float}
\usepackage{amsmath}
\usepackage{amssymb}
\usepackage{amsthm}
\usepackage{hyperref}
\usepackage{listings}
\usepackage{color}
\usepackage{algorithm}
\usepackage{algorithmic}
\usepackage{tikz}
\usepackage{pgfplots}
\usepackage{pgfplotstable}
\usepackage{caption}
\usepackage{subcaption}
\usepackage{multirow}
% 将字体改变为中文
\usepackage{ctex}
\usepackage{xeCJK}
\setCJKmainfont{SimSun}
\setCJKsansfont{SimHei}
\setCJKmonofont{SimSun}
\setCJKfamilyfont{zhsong}{SimSun}
\setCJKfamilyfont{zhhei}{SimHei}
\setCJKfamilyfont{zhkai}{KaiTi}
\setCJKfamilyfont{zhfs}{FangSong}
\setCJKfamilyfont{zhli}{LiSu}
\newcommand{\songti}{\CJKfamily{zhsong}} % 宋体
\newcommand{\heiti}{\CJKfamily{zhhei}} % 黑体
\newcommand{\kaishu}{\CJKfamily{zhkai}} % 楷书
\newcommand{\fangsong}{\CJKfamily{zhfs}} % 仿宋
\newcommand{\lishu}{\CJKfamily{zhli}} % 隶书
\usepackage{geometry}
\geometry{a4paper,scale=0.8}
\title{Result Analysis}
\begin{document}
\maketitle
\section{结论与讨论}
本研究通过对五个核心问题的分析与建模,获得了明确的量化结果(详见表格\ref{}),并得出以下主要结论:
\begin{itemize}
    \item \textbf{决策收敛性:} 在问题二的求解中,研究发现当用户行为模式稳定并遵循特定概率分布时,在100ms的决策周期内,系统决策会收敛于一个特定的最优策略。
    \item \textbf{关键影响因素:} 模型分析表明,用户请求的数据流量以及URLLC用户的服务质量(QoS)评价函数中的折扣因子$\alpha$是影响决策目标函数的关键变量。相比之下,其他因素(如系统噪声)的影响则不显著。
    \item \textbf{算法鲁棒性:} 通过对启发式函数进行变体测试,我们发现其形式对最终求解结果影响甚微。这证明了本研究采用的基于束搜索的近似贪心策略具有良好的鲁棒性。
\end{itemize}

基于以上分析,我们评估了该模型在实际应用中的潜力和局限性:
\begin{itemize}
    \item \textbf{适用场景:} 该模型特别适用于用户行为可预测且稳定的环境,例如工业自动化、智能楼宇等。在这些场景下,模型能在保障服务质量的同时,有效提升网络资源的利用效率。
    \item \textbf{个性化服务:} 模型充分考虑了不同用户类型在服务质量上的差异化需求,能够实现资源的按需分配,从而提高用户满意度。
    \item \textbf{求解效率:} 模型对同类型用户的排队情况进行了精细建模,并采用束搜索算法求解,能够在有限的计算资源下,高效地获得近似最优的决策方案。
    \item \textbf{局限性:} 该模型在用户行为模式发生剧烈变化的场景(如应急通信、自然灾害响应)中适用性较差,可能导致决策失效,无法满足关键时刻的用户需求。
    \item \textbf{核心假设:} 模型的有效性依赖于一个核心假设,即用户请求的数据流量在100ms决策周期内服从特定分布。在实际应用中,流量模式可能受时间、地点等多种外部因素影响,分布的动态变化可能削弱决策的有效性。
\end{itemize}

\section{未来工作展望}
为进一步提升模型的实用性与先进性,未来的研究可从以下方向展开:
\begin{itemize}
    \item \textbf{模型扩展:} 引入更多样化的用户类型和服务质量评价函数,以更精确地映射真实世界网络应用的多样化需求。
    \item \textbf{动态适应性:} 开发更具灵活性和自适应能力的决策算法,以应对用户行为的动态变化,特别是在突发事件和高动态场景下。
    \item \textbf{智能决策:} 融合机器学习与深度学习技术,特别是强化学习方法,以增强模型的环境感知、流量预测与自主决策能力。
    \item \textbf{实证验证:} 结合具体应用场景,开展仿真实验与物理测试,对模型的有效性、可靠性和性能进行全面的评估与验证。
    \item \textbf{算法优化:} 持续优化算法的计算复杂度与执行效率,以增强模型在大规模、高密度网络环境下的可扩展性与应用潜力。
\end{itemize}
\end{document}